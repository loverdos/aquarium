\documentclass[preprint,10pt]{sigplanconf}
\usepackage{amsmath}
\usepackage{amssymb}
\usepackage{graphicx}
\usepackage[british]{babel}
\usepackage{url}
\usepackage{listings}
\usepackage{color}
\usepackage{xspace}

\newcommand{\cL}{{\cal L}}
\newcommand{\TODO}{{\sl TODO \marginpar{\sl TODO}}}
\newcommand{\DTime}{\ensuremath{T}\xspace} % Time dimension
\newcommand{\DeltaDTime}{\ensuremath{\Delta{T}}\xspace}
\newcommand{\DUnitR}{\ensuremath{U_{R}}\xspace} % Unit of measure for resource R
\newcommand{\DeltaDUnitR}{\ensuremath{\Delta U_{R}}\xspace}

\begin{document}
\conferenceinfo{ScalaDays '12}{London, UK.}
\copyrightyear{2012}
\copyrightdata{1-59593-056-6/05/0006}

\titlebanner{DRAFT---Do not distribute}



\title{Aquarium: Billing for the Cloud in the Cloud}

\authorinfo{Georgios Gousios \and Christos KK Loverdos \and Nectarios Koziris}
{GRNET SA}
{\{gousiosg,loverdos,nkoziris\}@grnet.gr}

\maketitle
\begin{abstract}
    This paper describes the architecture for the Aquarium cloud infrastructure
    software. Aquarium is a new software we have built whose main function is to associate cloud resources usage with charging policies.
\end{abstract}

\category{D.2.11}{Software Architectures}{Domain-specific architectures}

\terms
    Algorithms, Performance

\keywords
    Billing, Cloud Computing, Scala, Akka

\section{Introduction}

Public cloud infrastructures have emerged as an alternative to building and
maintaining expensive proprietary data centers. At no upfront investment cost,
companies, especially startups, and individual users rent computing power from
service providers, using the pay-as-you-go charging model and enjoying high
scalability should the requirement arise~\cite{Lourid10}. The use of public cloud
infrastructures is particularly attractive for various forms of research
activities, where budget is usually tight and availability of computing power
can make or break an experiment. An important part of all public clouds is
resource management and customer billing~\cite{Armbr10}. Unfortunately, even
though all proprietary platforms do feature mechanisms for billing customers
for resource usage, there is currently a lack of open solutions.

In this paper we present Aquarium, an open source resource billing software,
designed for handling the production  requirements of {\sc grnet}'s public
Infrastructure as a Service (IaaS) platform. Aquarium utilizes a custom Domain
Specific Language for configuring the supported resources, the pricelists and
the billing algorithms. It receives input from an event queue and presents
billing results through a {\sc rest api}. It has been developed with Scala,
using the Akka library to handle concurrency and actor-based resource
processing. In the following sections, we present the requirements that have driven
Aquarium's design, the software architecture and implementation of important
computational algorithms and a preliminary evaluation of Aquarium's
performance. We also present an account of our experience in introducing Scala
in our company.

\section{Requirements}

Aquarium was designed on a clean sheet to serve a particular purpose,
namely the provision of billing services to an IaaS infrastructure,
and also be extensible to new services. In the following sections,
we briefly present the requirements that shaped Aquarium's design.

\subsection{Application Environment}
Aquarium developed as part of the Okeanos project at GRNet. The
Okeanos project is building a full stack public IaaS system for Greek
universities, and several services on top of it. Several components comprise
the Okeanos infrastructure:

\begin{description}

    \item[Synnefo] is an IaaS management console. Users can create and start
        VMs, monitor their usage, create private internal networks among VMs
        and connect to them over the web. The service backend is based on
        Google's Ganneti for VM host management and hundrends of physical
        VM container nodes.

    \item[Archipelago] is a storage service, based on the Rados
        distributed object store. It is currently under development, and the
        plan is to act as the single point of storage for VM images, shared
        volumes and user files, providing clonable snapshots and distributed
        fault tolerance.
    
    \item[Pithos] is a user oriented file storage service. Currently it its
        second incarnation, it supports content deduplication, sharing of files
        and folders and a multitude of clients.

    \item[Astakos] is an identity consolidation system that also acts as the
        entry point to the entire infrastructure. Users can login using 
        identities from multiple systems, such as the Shibboleth (SAML) 
        federation enabled across all Greek universities or their Twitter 
        accounts.

\end{description}

While all the above systems (and several prospective ones) have different 
user interfaces and provide distinct functionality in the context of
the GRnet IaaS, they all share a common notion of \emph{resources}, access
and manipulation options to which they offer to users. 

\subsection{Sharing}

The Okeanos IaaS will support the Greek higher education, an estimated
population of 100.000 students and researchers. Each member will be granted
access to a collection of resources using her institutional account. To enforce
a limit to the resources that can be acquired from the platform, each user will
have a limited amount of credits, renewable each month, which will be allowed
to spend on any resource available through the infrastructure. Resources can
also be shared; for example files on the Pithos file service or virtual machine
images on the Archipelago storage are potentially subject to concurrent usage from 
multiple users. This means that charges for the use of a single resource
may need to be distributed among several users. Also this may mean that in order
for sharing to work correctly, users may need to transfer credits among them.

\subsection{Resource Limiting}

The Okeanos IaaS will be free service for all users. As such, it is prone to
abuse, either by misappropriating resources or by utilizing resources for
purposes not related to research. A billing system can help with controlling
resource usage if it permits its users to use resources up to a periodically
renewable limit. In order facilitate the development of other infrastructure
system, this limit can be expressed into currency (real or provisional). To
support this scheme, Aquarium will need to update a user's balance almost
immediately after the user has claimed the required resources, and inform other
systems in case the user's credits are exhausted. 

\subsection{Configuration}

Billing systems are by nature open ended. As new services are deployed, new
resources appear, while others might be phased out.  Moreover, changes to
company policies may trigger changes to price lists for those resources, while
ad-hoc requests for large scale computational resources may require special
pricing policies. In order for a billing system to be able to successfully
adapt to changing requirements, it must be able to accommodate such changes
without requiring changes to the application itself. This means that all
information required for Aquarium in order to perform a billing operation,
must be provided to it externally. Moreover, to ensure high-availability,
billing configuration should be updatable while Aquarium is running, or at
least with minimal downtime, without affecting the operation of external
systems.


\subsection{Scaling}

In the context of the Okeanos system, Aquarium provides billing services on a
per user basis for all resources exposed by other systems. As such, it is in
the critical path of user requests that modify resource state; all supported
applications must query Aquarium in order to ensure that the user has enough
credits to create a new resource. This means that for a large number of users
(given previous GRNet systems usage by the Greek research community, we
estimate a concurrency level of 30.000 users), Aquarium must update and
maintain in a queryable form their credit status, 
with soft realtime guarantees. 

Being on the critical path also means that Aquarium must be highly resilient,
too. If Aquarium fails, all supported systems will also fail. Even if Aquarium
fails for a short period of time, it must not loose any billing events, as this
will allow users to use resources without paying for them. Moreover, in case of
failure, Aquarium must not corrupt any billing data under any circumstances,
while it should reach an operating state very fast after a service restart.

\section{Domain Modeling}

\subsection{Basic terminology}
We have already mentioned several entities in our description so far. Let us be a bit more specific on several key terms.

\begin{description}
\item[Credits]
The analog of money. Credits are the `universal money` within Aquarium.

\item[Resource]
A billable/chargeable entity. We generally need credits to use a resource. When a resource is used,  then consume credits. Examples of resources are the \textsf{download bandwidth} and, respectively,  the \textsf{upload bandwidth}, the \textsf{disk space} and the \textsf{VM time} to name a few. Generally speaking, the ``resource'' term specifies the type. A resource may have several properties attached, i.e. it name, unit of measure, a description of how its consumption translates to credit and whether it can have more than one instances.

\item[Resource instance]
A user may have several instances of a resource type. For example, regarding a ``Virtual Machine'' resource, a user may have more then one of them. They are distinguished by their unique resource instance identifier. We call resources that can have more than one instance ``complex'' resources.

\item[Resource event]
An event that is generated by a system, which is responsible for the resource.
The resource event describes a state change for the resource. In particular, a resource event records the time when that state change occurred and the changed value.

\item[Cost policy]
A cost policy refers to a resource and it is the policy used in order to charge credits for resource usage. Cost policies come in three flavors, namely \textsf{continuous}, \textsf{discrete} and \textsf{onoff}.
          
\item[Pricelists] assign a price tag to each resource, within a timeframe.

\item[Charging algorithms] specify the way a resource event can generate consumed credits. 
A charging algorithm can be as simple as a direct multiplication of the 
        chargeable resource quantity with the applicable price. We offer the option, though, to define more complex charging  scenarios, the Aquarium DSL supports a simple imperative language with
        a number of implicit variables (e.g. \texttt{price, volume, date}) 
        that enable administrators to specify, e.g. billing algorithms that
        scale with billable volume. Similarily to price lists, charging algorithms
        have an applicability timeframe attached to them.
        
\item[Credit plans] define a number of credits to give to users and a repetition
        period.

\item[Agreements] assign a name to a charging algorithm, pricelist and creditplan triplets,
        which is then assigned to each user.
        

\item[Resource instance state]
A value which is associated with a resource instance. Usually a floating point number, as in the $10.5$ MB designation regarding a possible ``total current downloading bandwidth''.

\item[User]
An owner of resources and credits. Users are defined externally of Aquarium.

\item[Resource event store]
A datatabase, in the general sense, where resource events are stored.

\item[User bill]
A, usually, periodic statement of how many credits the user has consumed. It may contain detailed analysis that relates consumed credits to resources.
  
\item[Billing period]
A time period at the end of which we issue a user bill.
A billing period is made of a starting date and a duration that is a multiple of a week.
A usual billing period starts on a particular month date (eg. 3rd) and lasts for a month.
Each resource type designates what happens to its accumulated value (if any) at the beginning of the billing period. Usually, at the beginning of the billing period, the accumulating amounts of resources are set accumulating amount. For example, for a monthly billing period, the total uploading bandwidth is reset to zero every month.
   
\item[User state]
The user state is made of the following distinct parts, of which the first two can be integrated to a unifying ``resource'' concept.

\begin{enumerate}
\item User credit state, that is the total credit amount for the user.

\item User resource state, which refers to the state of each resource instance that the user owns.

\item Processing state \TODO
\end{enumerate}

\item[Resource event processing]
The set of algorithmic steps by which a resource event leads to state changes of the user state.

\end{description}

\subsection{Representation of events}
Aquarium is the recipient of several types of events from external systems. More specifically, systems that manage the lifetime and operation of chargeable resources are responsible to send Aquarium events that describe notable resource state changes. As an example, when a user consumes disk space, Pithos, which is the service responsible for user storage, will send a resource event for the \textsf{diskspace} resource and the amount of bytes used.

Another type of events are the user-related events, from the perspective if the Identity Management service. For example, Aquarium needs to know when a user is first created and when the user is activated or suspended. Also, several types of users (e.g. Lab Administrators or Professors in the academic setting where Aquarium is initially targeted for) may be assigned more privileged credit plans that periodically give them more credits.

\begin{figure}
\lstset{language=c, basicstyle=\footnotesize,
stringstyle=\ttfamily, 
flexiblecolumns=true, aboveskip=-0.9em, belowskip=0em, lineskip=0em}

\begin{lstlisting}
abstract class AquariumEvent(
  val id: String,
  val occurredMillis: Long,
  val receivedMillis: Long)
  
case class ResourceEvent(
    override val id: String,
    override val occurredMillis: Long, 
    override val receivedMillis: Long, 
    userId: String,
    clientId: String,               
    resource: String,
    value: Double,
    details: Map[String, String])
  extends AquariumEvent(id, occurredMillis, receivedMillis)
\end{lstlisting}
\caption{Representation of events in Scala.}
\label{fig:aqevent}
\end{figure}

The actual format of the event is presented in Figure~\ref{fig:resevt}.

\begin{figure}
\lstset{language=C, basicstyle=\footnotesize,
stringstyle=\ttfamily, 
flexiblecolumns=true, aboveskip=-0.9em, belowskip=0em, lineskip=0em}

\begin{lstlisting}
{
  "id":"4b3288b57e5c1b08a67147c495e54a68655fdab8",
  "occuredMillis":1314829876295,
  "receivedMillis":1314829876300,
  "userId": "31",
  "cliendId": "snf-astakos-1",
  "resource": "vmtime",
  "value": 1,
  "details": {
    "vmid": "3300",
    "action": "on"
  }
}
\end{lstlisting}
\caption{A JSON-formatted \texttt{ResourceEvent}} 
\label{fig:resevt}
\end{figure}

For the exchange of events we adopt the ubiquitous JSON format. This is chosen for the ease of manipulation from both Scala, used in Aquarium, and Python, used for the rest of the communicating systems. Our base entity, \texttt{AquariumEvent} and the \texttt{ResourceEvent} corresponding to a resource event, as described previously, are shown in Figure~\ref{fig:aqevent}, while Figure~\ref{fig:resevt} presents a JSON-formatted \texttt{ResourceEvent} value. The given attributes are:

\begin{description}
\item[id] is the event \textsf{ID} at the client side, that is the sender of the event. Aquarium requires that \textsf{IDs}

\item[occurredMillis] gives us the time of when the even occurred, using the Unix time convention.

\item[receivedMillis] gives us the time of when the event was received in the Aquarium processing pipeline, also using the Unix time convention.

\item[userId] is the unique identifier of the user this event relates to. Use identifiers are managed by the Identity Management system, Astakos.

\item[clientId] is a unique, across all systems, identifier for the external system generating the event.

\item[resource] is the resource name.

\item[value] is the resource's changed value.

\item[details] is a resource-specific collection of  attributes and their values.
\end{description}

Of particular interest is the \textsf{details} map. Several resource types can use this map in order to pass resource-specific attributes to Aquarium, which in turn can be hooked into the charging algorithm. We motivate this extension mechanism with an example for the \textsf{vmtime} resource type, which simply represents usage of virtual machines. A respective resource event needs to specify:
\begin{itemize}
\item Which particular resource instance (which virtual machine) it refers to.
\item The relevant state change for the resource instance. In this case, a virtual machine can be started (\textsf{on} state) or stopped (\textsf{off} state).
\end{itemize}

In our example of Figure~\ref{fig:resevt}, we use the \textsf{vmId} and \textsf{action} extensions to specify that the virtual machine instance with identifier \textsf{3300} was started, that is it transitioned to \textsf{on} state. We should generally note that each resource type is free to choose the domain-specific description for the instance \textsf{ID}. 

Finally, for the timing of events we assume all systems that send event to Aquarium have synchronized clocks. We actually \textit{require} this, so that Aquarium is not concerned with time book keeping.


\subsection{The configuration DSL}
\label{sec:dsl}
We addressed the configuration requirements of Aquarium by creating a new
domain specific language ({\sc dsl}), based on the YAML format.  The DSL
enables administrators to specify chargeable resources, charging policies and
price lists and combine them arbitrarily into agreements applicable to specific
users, user groups or the whole system. 
The DSL supports inheritance for policies, price lists and agreements and composition in the case of agreements.
It also facilitates the
definition of generic, repeatable debiting rules, which are then used by the
system to refill the user's account with credits on a periodic base.

From the previously specified term, the following five are used in the DSL:

\begin{enumerate}
\item Resources
\item Charging algorithms
\item Price lists
\item Credit plans
\item Agreements
\end{enumerate}


\begin{figure}
\lstset{language=c, basicstyle=\footnotesize,
stringstyle=\ttfamily, 
flexiblecolumns=true, aboveskip=-0.9em, belowskip=0em, lineskip=0em}

\begin{lstlisting}
resources:
  - resource:
    name: bandwidthup
    unit: MB/hr
    complex: false
    costpolicy: continuous
pricelists:
  - pricelist: 
    name: default
    bandwidthup: 0.01
    effective:
      from: 0
  - pricelist: 
    name: everyTue2
    overrides: default
    bandwidthup: 0.1
    effective:
      repeat:
      - start: "00 02 * * Tue"
        end:   "00 02 * * Wed"
      from: 1326041177        //Sun, 8 Jan 2012 18:46:27 EET
algorithms:
  - algorithm:
    name: default
    bandwidthup: $price times $volume
    effective:
      from: 0
agreements:
  - agreement:
    name: scaledbandwidth
    pricelist: everyTue2
    algorithm:
      bandwidthup: |
        if $volume gt 15 then
          $volume times $price
        elsif $volume gt 15 and volume lt 30 then
          $volume times $price times 1.2
        else
          $volume times price times 1.4
        end
\end{lstlisting}

\caption{A simple billing policy definition.} 
\label{fig:dsl}
\end{figure}

In Figure~\ref{fig:dsl}, we present the definition of a simple (albeit valid) 
policy. The policy parsing is done top down, so the order of definition 
is important. The definition starts with a resource, whose name is then
re-used in order to attach a pricelist and a price calculation algorith to it.
In the case of pricelists, we present an example of \emph{temporal overloading};
the \texttt{everyTue2} pricelist overrides the default one, but only for 
all repeating time frames between every Tuesday at 02:00 and Wednesday at
02:00, starting from the timestamp indicated at the \texttt{from} field. Another
example of overloading is presented at the definition of the agreement, which
overloads the default algorithm definition using the imperative part of the
Aquarium {\sc dsl} to provide a scaling charge algorithm.

\section{Architecture}
\paragraph{Architectural decisions} 

Aquarium's architectural design is driven by two requirements: scaling
and fault tolerance. Although initially we used a 3-tiered
architecture, it quickly became clear that it would not meet our
needs. Complications arose from the difficulty of describing versioned
tree-based structures, such as the configuration {\sc dsl}, in a
relational format, and in making sure that resource events were
described in an abstract way that would be adaptable to all future
system expansions. Moreover, for performance reasons, Aquarium must
maintain in-memory caches of computed values; for example the single
query that cloud services will be asking Aquarium continually (number
of remaining credits) must be answered within a few milliseconds for a
large number of concurrent requests. Consequently, Aquarium's data
processing architecture was based on the event sourcing
pattern~\cite{Fowle05}, while system state handling and processing
components are modeled as collections of actors~\cite{Hewit73}.

Event sourcing assumes that all changes to application state are
stored as a sequence of events, in an immutable log. With such a log,
Aquarium can rebuild its state at any point in time by replaying the
events in order, so it is possible to perform queries on past system
states for debugging purposes. Similarly, Aquarium can concurrently
employ several event processing models to cater for different
front-end data requirements. Furthermore, application crashes are not
destructive for Aquarium, as long as event replay is fast enough and
no state is inserted to the application without being recorded to the
event log first.

\begin{figure}
    \begin{center}
    \includegraphics[scale=1.2]{arch.pdf}
    \end{center}
\caption{Functional components in Aquarium's architecture} 
\label{fig:arch}
\end{figure}

We use actors to encapsulate state. The actor model guarantees that
only one thread touches the actor state, thus eliminating the need for
locks.

\paragraph{Components} An overview of the Aquarium architecture is
presented in Figure~\ref{fig:arch}. The system is modeled as a
collection of logically and functionally isolated components that
communicate by message passing. Within each component, a number of
actors take care of concurrently processing incoming messages through
a load balancer component that is the gateway to requests targeted to
the component. Each component is also monitored by its own supervisor
(also an actor); should an actor fail, the supervisor will
automatically restart it. The architecture allows certain application
paths to fail individually while the system is still responsive, while
also enabling future distribution of multiple components on clusters
of machines.

The system receives input mainly from two sources: queues for
resource and user events and a {\sc rest api} for credits and resource
state queries. The queue component reads messages from a configurable
number of queues and persists them in the application's immutable log
store. Both input components then forward incoming messages to a
network of dispatcher handlers which do not do any processing by
themselves, but know where the user actors lay. As described earlier,
actual processing of billing events is done within the user actors.
Finally, a separate network of actors take care of scheduling periodic
tasks, such as refiling of user credits; they do so by issuing events
to the appropriate queue.

\paragraph{Implementation}

Aquarium is being developed as a standalone service, based on the Akka
library for handling actor related functionality. Akka also provided
actor-based components for communicating with the message queue and,
through a third party component (Spray), facilities for handling {\sc
  rest} requests. We chose the {\sc amqp} protocol and its Rabbit{\sc
  mq} implementation for implementing the request queue because recent
versions include support for active/active cluster configurations. The
persistence layer is currently implemented by Mongo{\sc db}, for its
replication and sharding support. However, this is not a hard
requirement, as Aquarium features an abstraction layer for all
database queries (currently 10 methods), which can then be implemented
by any persistence system, relational or not.


\section{Computational aspects}

\subsection{Charging basics}
The previously mentioned DSL gives us a declarative way of describing the charging algorithms and the other related concepts. The computational engine of Aquarium then transforms these declarations to computing steps for the charging process.

Charging, based on a resource event, is inherently multidimensional. Time (\DTime) is always one dimension to consider: there are resources whose charging is based on the passing of time. Also, rather obviously, the unit of measure (\DUnitR) for a resource ($R$) is yet another dimension to take into account. These dimensions usually enter the formulas of the charging algorithms, although it is not necessary that all of them do. But there is one more aspect to consider, which clearly adds up to the multidimensionality of the problem. More specifically, \DUnitR can be taken into account either as whole value or as a difference \DeltaDUnitR. An avid reader may be already wondering about time. Shouldn't it be treated in the same footing, at least for reasons of conceptual uniformity? Shouldn't we treat both \DTime and \DeltaDTime? In reality, we \textit{always} work with time differences. Usually, \DeltaDTime means the time difference between two resource events for the same resource instance. In such a setting, \DTime would mean the time difference from some well-known and predefined point in time. This is generally a possibility, for example by considering the beginning of a billing period as the beginning of time as far as a particular charging calculation is concerned. So all combinations of either full values or differences can be considered, as given in Table~\ref{tab:dt}.

\begin{table}[htdp]
\label{tab:dt}
\begin{center}
\begin{tabular}{|c|c|c|}
\hline
&Time & Resource unit of measure \\
\hline
Absolute value & \DTime & \DUnitR \\
Difference & \DeltaDTime  & \DeltaDUnitR \\
\hline
\end{tabular}
\end{center}
\label{default}
\caption{Considering time and resource unit of measure as absolute values or differences
}
\end{table}%


Now, we can take this reasoning a bit further, by 

Per resource, the charging operation is affected by the cost policy and complexity
parameters. Specifically, the 3 available cost policies affect the calculation 
of the amount of resource usage to be charged as follows:

\begin{itemize}
    \item resources employing the \textsf{continuous} cost policy are charged for
        the actual resource usage through time. When a resource event arrives,
        the previous resource state between the previous charge operation and the
        current event event timestamp is charged and the resource state is then
        updated. More formally, for continuous resources, if $f(t)$ represents
        the function of resource usage through time and $p(t)$ is the function
        representing the pricelist at time $t$, 
        then the total cost up to a 
        $c(t) = \sum_{i=0}^{t} {p(t) \times \int_0^{t}{f(t)dt}}$. Most resources
        in Aquarium are continuous, for example bandwidth and disk space.

    \item resources employing the \textsf{onoff} cost policy can be in two states:
        either switched on and actively used or switched off. Therefore, the unit
        of resource usage is time and not the actual resource usage, while the
        period of charging is calculated only when the resource is switched on.
        Virtual machines are examples of resources with the \textsf{onoff} cost
        policy.

    \item resources using the \textsf{distinct} cost policy are charged
        upon usage, without time playing a role in the charge. Such resources
        are useful for one off charges, such as the allocation of
        virtual machine or the migration of a virtual machine to a less busy
        host.

\end{itemize}


\subsection{State management}
Commonly to most similar systems, billing in Aquarium is the application of the
provisions of a user's contract to an incoming billing event in order to
produce an entry for the user's wallet. However, in stark contrast to most
other systems, which rely on database transactions in order to securely modify
the user's balance, Aquarium performs account updates asynchronously and
concurrently for all users.

Billing events are obtained through a connection to a message queue. Upon
arrival, a billing event is stored in an immutable log, and then forwarded to
the user actor's mailbox; the calculation of the actual billing entries to be
stored in the user's wallet is done within the context of the user actor,
serially for each incoming events. This permits the actor to have mutable state
internally (as described in Section~\ref{sec:ustate}), without risking the
calculation correctness. The calculation process involves steps such as
validating the resource event, resolving the current state of resource affected
by the incoming resource event, deciding the value applicable pricelist and
algorithm, generating entries for the user's wallet and updating the current
resource state for the user. A significant source of complexity in the process
is the support for temporal overriding for pricelists and algorithms: within
the timeframe between resource updates, several policies or algorithms may be
active. The billing algorithm must therefore split the billing period to pieces
according the applicability of each policy/algorithm and make sure that at 
least a baseline policy is in effect in order to perform the calculation.
Consequently, a resource event might lead to several entries to the user's wallet.

\subsection{User State}
\label{sec:ustate}

\section{Performance}

To evaluate the performance of Aquarium, we formulated an experiment that
evaluated two important properties: the time required to perform the charging
operation for a resource event and the overall time required to process a
resource event, end to end. To conduct the experiment, Aquarium was configured,
using the policy {\sc dsl} to handle billing events for 5 types of resources,
using 3 overloaded pricelists, 2 overloaded algorithms, all of which were
combined to 5 different agreements. Aquarium's data store was pre-filled in
with 1.000.000 resource events, evenly distributed among 1.000 users. To drive
the benchmark, we used a synthetic load generator that produced random
billing events, at a configurable rate per minute. 

To run the benchmark, we deployed Aquarium on a virtualized 4 core 2{\sc gh}z
class {\sc cpu} and 4{\sc gb} {\sc ram} Debian Linux server, on the Okeanos
cloud infrastructure. The virtual machine running Aquarium was configured with
a 4{\sc gb} maximum heap size. We selected Rabbit{\sc mq} and Mongo{\sc db} as
the queue and database server respectively, both of which where run in another
4-core, 4{\sc gb ram} virtual machine. Both systems were run using current
versions at the time of benchmarking (2.7.1 for Rabbit{\sc mq} and 2.6 for
Mongo{\sc db}). The two virtual machines did not share a physical host and
communicated over Okeanos's switched network fabric at an effective rate of 800
Mbits/sec, as reported by the \texttt{iperf} utility.  We paid particular
attention to have Mongo{\sc db} load the full working data set in memory, by
repeatedly querying all stored records. No further optimization was performed
on either back-end system.

\begin{figure}[t]
    \begin{center}
        \includegraphics[scale=0.63]{perf.pdf}
    \end{center}

    \caption{Average time for peforming a billing operation and end for end to
    end message processing for 100 active users and a varying number of
    messages per minute.}
    
    \label{fig:perf}
\end{figure}

All measurements were done using the first working version of the Aquarium
deployment, so no real optimisation effort has taken place. This shows in the
current performance measurements, as Aquarium was not able to handle more that
about 500 billing operations per second. One factor that contributed to this
result was the way resource state recalculations was done; in the current
version, the system needs to re-read parts of the event and billing state from
the database every time a new resource event appears. This contributes to more
than 50\% of the time required to produce a charging event, and can be
completely eliminated when proper billing snapshots are implemented. In other
measurements, we also observed that the rate of garbage creation was extremely
high, more that 250 {\sc mb/sec}. Upon further investigation, we attributed it
to the way policy timeslot applicability is calculated. Despite the high
allocation rate, the {\sc jvm}'s garbage collector never went through a
full collection cycle; when we forced one after the benchmark run was over, 
we observed that the actual heap memory usage was only 80{\sc mb}, which
amounts to less than 1 {\sc mb} user.

Even so, by extrapolating on the results and hardware configuration, an average
12-core box could handle more 1.500 messages per minute from about 300 active
users, at 5 events per minute. Given that activity from users is expected to
arrive in bursts, a more realistic expectation might be to receive 1 message
per user per 5 minutes on average; in that case, and provided that the resource
query cost is negligible as it is being served from an in memory cache, the
system could handle around 4.500 concurrent users. While such back of the
envelop calculations do not account for traffic spikes, they do provide an
rough estimation of the optimisation effort that must be put in place. 


\section{Lessons Learned}

One of the topics of debate while designing Aquarium was the choice of
programming platform to use. With all user facing systems in the Okeanos cloud
being developed in Python and the initial Aquarium designers being beginner
Scala users (but experts in Java), the choice certainly involved risk that
management was initially reluctant to take. However, by breaking down the
requirements and considering the various safeguards that the software would
need to employ in order to satisfy them, it became clear that a
typesafe language was a hard requirement. Of the platforms examined, the {\sc
jvm} had the richest collection of ready made components; the Akka library was
particularly enticing for the scalability and distribution possibilities it
offered.

The choice of Scala at the moment it had been made was a high risk/high gain
bet for GRNet. However, the development team's experience has been generally
positive. Scala as a language was an enabling factor; case classes permitted
the expression of data models, including the configuration {\sc dsl}, that
could be easily be serialized or read back from wire formats while also
promoting immutability through the use of the \texttt{copy()} constructor. The
pervasive use of immutability allowed us to write strict, yet simple and
concise unit tests, as the number of cases to be examined was generally low.
The \textsf{Maybe}\footnote{\textsf{Maybe} works like \textsf{Option}, but it
has an extra possible state (\textsf{Failed}), which allows exceptions to be
encapsulated in the return type of a function, and then retrieved and accounted
for in a pattern matching operation with no side effects. More at
\url{https://github.com/loverdos/Maybe}} monad, enabled side-effect free
development of data processing functions, even in cases where exceptions were
the only way to go. Java interoperability was excellent, while thin Scala
wrappers around existing Java libraries enabled higher productivity and use of
Scala idioms in conjunction with Java code.

The Akka library, which is the backbone of our system, is a prime example of 
the simplicity that can be achieved by using carefully designed high-level
components. Akka's custom supervision hierarchies allowed us to partition the
system in self-healing sub-components, each of which can fail independently
of the other. For example, if the queue reader component fails due to a queue
failure, Aquarium will still be accessible and responsive for the {\sc rest}
interface. Also, Akka allowed us to easily saturate the processing components
of any system we tested Aquarium on, simply by tuning the number of threads (in
{\sc i/o} bound parts) and actors (in {\sc cpu} bound parts) per dispatcher. 

Despite the above, the experience was not as smooth as initially expected. The
most prominent problem we encountered was that of lacking documentation. The
Akka library documentation, extensive as is, only scratches the surface.
Several other libraries we use, for example Spray for {\sc rest} handling, have
non-existent documentation. The Java platform, and .Net that followed, has
shown that thorough and precise documentation are key to adoption, and we
expected a similar quality level. Related is the problem of shared community
wisdom; as most developers know, a search for any programming problem will
reveal several straightforward Java or scripting language sources. The
situation with Scala is usually the opposite; the expressive power of the
language makes it the current language of choice for treating esoteric
functional programming concepts, while simple topics are often neglected. Scala
has several libraries of algebraic datatypes but no {\sc yaml} parser. As Scala
gains mainstream adoption, we hope that such problems will fade.

From a software engineering point of view, the current state of the project was
reached using about 6 person months of effort, 2 of which were devoted to
requirements elicitation, prototype building and familiarizing with the
language. The source code currently consists of 5.000 lines of executable
statements (including about 1.000 lines of tests), divided in about 10
packages. The system is built using both {\sc sbt} and Maven. 

\section{Related Work}

\section{Conclusions and Future Work}
In this paper, we presented Aquarium, a high-performance, generic billing 
system, currently tuned for cloud applications. We presented the requirements
that underpinned its design, outlined the architectural decisions made
and analysed its implementation and performance.

Scala has been an enabling factor for the implementation of Aquarium, both
at the system prototyping phase and during actual development. 

Aquarium is still under development, with a first stable version 
being planned for early 2012. 

Aquarium is available under an open source license at 
\url{https://code.grnet.gr/projects/aquarium}.

\bibliographystyle{abbrvnat}
\bibliography{aquarium}

\end{document}
