\documentclass[letterpaper,twocolumn,10pt]{article}
\usepackage{usenix, epsfig, endnotes, xspace, url}

\usepackage{amsmath}
\usepackage{amssymb}
\usepackage{graphicx}
\usepackage{listings}
\usepackage{color}

\newcommand{\cL}{{\cal L}}
\newcommand{\TODO}{{\sl TODO \marginpar{\sl TODO}}}
\newcommand{\DTime}{\ensuremath{T}\xspace} % Time dimension
\newcommand{\DeltaDTime}{\ensuremath{\Delta{T}}\xspace}
\newcommand{\DUnitR}{\ensuremath{U_{R}}\xspace} % Unit of measure for resource R
\newcommand{\DeltaDUnitR}{\ensuremath{\Delta U_{R}}\xspace}
\newcommand{\MB}[1]{\ensuremath{#1\,\mbox{\sc MB}}}
\newcommand{\GB}[1]{\ensuremath{#1\,\mbox{\sc GB}}}
\newcommand{\secs}[1]{\ensuremath{#1\,sec}}

\newcommand{\todo}[1]{\textbf{TODO}\footnote{\textbf{TODO:} #1}}

\newcommand{\grnet}{{\sc grnet}\xspace}

\begin{document}

%don't want date printed
\date{}

%make title bold and 14 pt font (Latex default is non-bold, 16 pt)
\title{Aquarium: Billing for the Cloud in the Cloud}

\author{
{\rm Georgios Gousios}\\
\and
{\rm Christos KK Loverdos}\\
\and
{\rm Panos Louridas}\\
\and
{\rm Nectarios Koziris}\\
\and
\multicolumn{1}{p\textwidth}{\centering Greek Research and Technology Network (GRNET)}\\
% copy the following lines to add more authors
% \and
% {\rm Name}\\
%Name Institution
} % end author

\maketitle


% Use the following at camera-ready time to suppress page numbers.
% Comment it out when you first submit the paper for review.
\thispagestyle{empty}

\subsection*{Abstract}

An important part of public IaaS offerings is resource management and
customer billing. In this paper we present the design and
implementation of Aquarium, an extensible billing service software.
Aquarium associates state changes in cloud resources with respective
charges, based on configurable, user-specific and versioned charging
policies. The implementation of Aquarium is characterized by pervasive
data immutability, actor message passing, and service orientation.

\section{Introduction}

Public cloud infrastructures have emerged as an alternative to
building and maintaining expensive proprietary data
centers~\cite{Lourid10}. An important part of all public clouds is
resource management and customer billing~\cite{Armbr10}. Even though
all proprietary platforms feature mechanisms for billing customers for
resource usage, there is currently a lack of open solutions.

In this paper we present Aquarium, an open source resource billing software,
designed to handle the production  requirements of \grnet's public
Infrastructure as a Service (IaaS) platform. Aquarium utilizes a custom Domain
Specific Language ({\sc dsl}) for configuring the supported resources, the
price lists, and the billing algorithms. It receives input from an event queue
and presents billing results through a {\sc rest api}. It has been developed
in Scala, using the Akka library to handle concurrency and actor-based
event processing. In the following sections we present the requirements
that have driven Aquarium's design, the software architecture and
implementation of important computational algorithms, and a preliminary
evaluation of Aquarium's performance. 

\section{Requirements}

Aquarium is developed to provide accounting and billing to \grnet's
cloud computing services---although its design ensures that it can be
used by other cloud infrastructures. These services include provision
of {\sc vm}s and online storage; additional services will be built on
top of them (for instance, repository services and services for big
data computations). Aquarium should be therefore able to:

\begin{itemize}

\item Provide accounting and billing for different resources, not all
  of them known in advance.

\item Allow the application of different pricing policies to different
  resources, for different users, at different periods of time.

\item Allow for dynamic modification of any of the above, while
  maintaining full traceability of changes.

\item Provide a consistent (not just eventually consistent) view of
  users' resource usage and billing while scaling to thousands of
  concurrent users.

\end{itemize}

As Aquarium must accommodate different resources with different
policies for different users over time, we were faced either with a
logistical nightmare of giving users resource chunks directly, or
adopting a monetary approach, where users are given credit that they
can use as they see fit. While in \grnet's case users are not charged
real money for using the services, so the credit is virtual, this
should not make any difference to Aquarium. \grnet will supply virtual
currency to its users, which they will be able to spend on acquiring
and using resources. If real cash were used, \grnet (or any entity
using Aquarium) would only need to substitute real credit for virtual
credit, without any changes in Aquarium itself.

Aquarium is in the critical path of user requests that modify resource
state; all supported applications must query Aquarium in order to
ensure that the user has enough credits to create a new resource. This
means that for a large number of users (in the order of tens of thousands),
Aquarium must update and maintain in a queryable form their credit
status, with soft real time guarantees.

Being on the critical path also means that Aquarium must be highly
resilient. If Aquarium fails, even for a short period of time, it must
not loose any billing events, as this will allow users to use
resources without being charged. Moreover, in case of failure,
Aquarium must not corrupt any billing data, while it should reach an
operating state very fast after a service restart.

\section{Resources, Events, Policies and Cost Calculation}

A resource represents an entity that can be charged for. Aquarium does
not assume a fixed set of resource types and is extensible to any
number of resources. A resource is associated with a name and a cost
unit. Depending on whether a resource can have many instances per
user, a resource can be complex or simple. Finally, each resource is
associated with a cost calculation policy.

Usage of resources by users of external systems triggers the generation or
resource events, which are received by Aquarium and charged for on a per user
basis. As an example, after a successful file upload to a cloud storage
service, a resource event for the \textsf{diskspace} resource along with the
amount of bytes consumed will be sent to Aquarium.  Figure~\ref{fig:resevt}
presents an example of a {\sc json}-formatted resource event.  Apart from
resource events, Aquarium also processes events relating to users
(e.g., user creation).

\begin{figure}
\lstset{language=C, basicstyle=\footnotesize,
stringstyle=\ttfamily, 
flexiblecolumns=true, aboveskip=-0.9em, belowskip=0em, lineskip=0em}

\begin{lstlisting}
{ "id":"4b3288b57e5c1b08a67147c495e54a68655fdab8",
  "occuredMillis":1314829876295,
  "receivedMillis":1314829876300,
  "userId": "31",
  "cliendId": "pithos-storage-service",
  "resource": "diskspace.1",
  "value": 10,
  "instance-id" : 3300 
}
\end{lstlisting}
\caption{A JSON-formatted \texttt{ResourceEvent}} 
\label{fig:resevt}
\end{figure}

The cost calculation engine in Aquarium is configured by a 
custom {\sc dsl}, based on the {\sc yaml} format. The {\sc dsl} enables
us to specify resources, charging algorithms and
price lists and combine them arbitrarily into agreements applicable to
specific users, user groups or the whole system. It
supports inheritance for policies, price lists and agreements and
composition in the case of agreements. It also facilitates the
definition of generic, repeatable debiting rules, that specify
periodically refills of users credits.

In Figure~\ref{fig:dsl}, we present the definition of a simple but
valid policy. Policy parsing is done top down, so the order of
definition is important. The definition starts with a resource, whose
name is then re-used when attaching a price list and a charging
algorithm to it. In the case of price lists, we present an example of
\emph{temporal overloading}; the \texttt{everyTue2} pricelist
overrides the default one, but only for all repeating time frames
between every Tuesday at 02:00 and Wednesday at 02:00, starting from
the timestamp indicated at the \texttt{from} field.

\begin{figure}
\lstset{language=c, basicstyle=\footnotesize,
stringstyle=\ttfamily, 
flexiblecolumns=true, aboveskip=-0.9em, belowskip=0em, lineskip=0em}

\begin{lstlisting}
resources:
  - resource:
    name: bandwidthup
    unit: MB/hr
    complex: false
    costpolicy: continuous
pricelists:
  - pricelist: 
    name: default
    bandwidthup: 0.01
    effective:
      from: 0
  - pricelist: 
    name: everyTue2
    overrides: default
    bandwidthup: 0.1
    effective:
      repeat:
      - start: "00 02 * * Tue"
        end:   "00 02 * * Wed"
      from: 1326041177        //Sun, 8 Jan 2012 18:46:27 EET
algorithms:
  - algorithm:
    name: default
    bandwidthup: $price times $volume
    effective:
      from: 0
agreements:
  - agreement:
    name: scaledbandwidth
    pricelist: everyTue2
    algorithm: default
\end{lstlisting}
%$
\caption{A simple billing policy definition.} 
\label{fig:dsl}
\end{figure}

%\section{Computational aspects}
%\label{computation}

\paragraph{Cost calculation}In order to charge based on the incoming
resource events, time (\DTime) and the unit of measure (\DUnitR) for a
resource ($R$) play a central role. Below, we present charging
scenarios for three well-known resources, namely \textsf{bandwidth},
\textsf{diskspace} and \textsf{vmtime}. For the analysis of each case,
we assume:

\begin{itemize}
\item The arrival of two consecutive resource events, happening at
  times $t_0$ and $t_1$, with a time difference of
  $\Delta T = t_1 - t_0$.

\item The total values of resource $R$ for times $t_0$ and $t_1$ are
  $U_R^0$ and $U_R^1$ respectively.

\item A ratio of the form $[\frac{C}{D}]$ represents the
  resource-specific charging unit, where $C$ is the credit unit, and
  $D$ depends on the combination of the previously discussed
  dimensions ($T$, $U_R$) that enters the calculation. The meaning of
  the factor is ``credits per $D$''.
\end{itemize}

\paragraph{\textsf{bandwidth}}
In this case, an event at $t_1$ records a change of bandwidth, using
the relevant unit of measure $U_R$. The credit usage computation is
$\Delta U_R \cdot [ \frac{C}{U_R} ]$. For example, let $\Delta U_R =
\MB{10}$; then the bandwidth charging unit $[ \frac{C}{U_R} ]$ is
``$credits$ per {\sc MB}'', since $U_R$ is measured in {\sc MB}.

\paragraph{\textsf{diskspace}}
We take into account the disk space $U_R^0$ occupied at $t_0$ together
with the time passed, $\Delta T$. The credit usage computation is
$U_R^{0} \cdot \Delta T \cdot [ \frac{C}{U_R \cdot T} ]$. That is,
when we receive a new state change for disk space, we calculate for
how long we occupied the total disk space without counting the new
state change. If we had \GB{1} at $t_0 = \secs{1}$ and we gained
another \GB{3.14} at $t_1 = \secs{3.5}$ then we are charged for the
\GB{1} we occupied for $3.5 - 1 = 2.5$ seconds. The disk space
charging unit $[ \frac{C}{U_R \cdot T} ]$ is ``$credits$ per {\sc GB}
per $sec$'', assuming $U_R$ (disk space) is measured in {\sc GB} and
time in $sec$.

\paragraph{\textsf{vmtime}}
Events for VM usage come into pairs that record \textsf{on} and
\textsf{off} states of the VM. We use the time difference between
these events for the credit usage computation, given by $\Delta T
\cdot [ \frac{C}{T} ]$.

The charging algorithms for the sample resources given previously
motivate related cost policies, namely \textsf{discrete},
\textsf{continuous} and \textsf{onoff}. Resources employing the
\textsf{discrete} cost policy are charged just like
\textsf{bandwidth}, those employing the \textsf{continuous} cost
policy are charged like \textsf{diskspace} and finally resources with
a \textsf{onoff} cost policy are charged like \textsf{vmtime}. Due to
space limits we omit a more detailed analysis and the description of
more involved scenarios.

Scheduled tasks compute total charges, the updated resource state and
a total credit amount for each billing period. This computation is
recorded in a persistent store, in an append-only fashion, for future
reference and as a cached value. This cached value can be handy in
computations or in system crashes, to avoid recomputation based on the
whole history of events.

\section{Architecture}
\paragraph{Architectural decisions} 

Aquarium's architectural design is driven by two requirements: scaling
and fault tolerance. Although initially we used a 3-tiered
architecture, it quickly became clear that it would not meet our
needs. Complications arose from the difficulty of describing versioned
tree-based structures, such as the configuration {\sc dsl}, in a
relational format, and in making sure that resource events were
described in an abstract way that would be adaptable to all future
system expansions. Moreover, for performance reasons, Aquarium must
maintain in-memory caches of computed values; for example the single
query that cloud services will be asking Aquarium continually (number
of remaining credits) must be answered within a few milliseconds for a
large number of concurrent requests. Consequently, Aquarium's data
processing architecture was based on the event sourcing
pattern~\cite{Fowle05}, while system state handling and processing
components are modeled as collections of actors~\cite{Hewit73}.

Event sourcing assumes that all changes to application state are
stored as a sequence of events, in an immutable log. With such a log,
Aquarium can rebuild its state at any point in time by replaying the
events in order, so it is possible to perform queries on past system
states for debugging purposes. Similarly, Aquarium can concurrently
employ several event processing models to cater for different
front-end data requirements. Furthermore, application crashes are not
destructive for Aquarium, as long as event replay is fast enough and
no state is inserted to the application without being recorded to the
event log first.

\begin{figure}
    \begin{center}
    \includegraphics[scale=1.2]{arch.pdf}
    \end{center}
\caption{Functional components in Aquarium's architecture} 
\label{fig:arch}
\end{figure}

We use actors to encapsulate state. The actor model guarantees that
only one thread touches the actor state, thus eliminating the need for
locks.

\paragraph{Components} An overview of the Aquarium architecture is
presented in Figure~\ref{fig:arch}. The system is modeled as a
collection of logically and functionally isolated components that
communicate by message passing. Within each component, a number of
actors take care of concurrently processing incoming messages through
a load balancer component that is the gateway to requests targeted to
the component. Each component is also monitored by its own supervisor
(also an actor); should an actor fail, the supervisor will
automatically restart it. The architecture allows certain application
paths to fail individually while the system is still responsive, while
also enabling future distribution of multiple components on clusters
of machines.

The system receives input mainly from two sources: queues for
resource and user events and a {\sc rest api} for credits and resource
state queries. The queue component reads messages from a configurable
number of queues and persists them in the application's immutable log
store. Both input components then forward incoming messages to a
network of dispatcher handlers which do not do any processing by
themselves, but know where the user actors lay. As described earlier,
actual processing of billing events is done within the user actors.
Finally, a separate network of actors take care of scheduling periodic
tasks, such as refiling of user credits; they do so by issuing events
to the appropriate queue.

\paragraph{Implementation}

Aquarium is being developed as a standalone service, based on the Akka
library for handling actor related functionality. Akka also provided
actor-based components for communicating with the message queue and,
through a third party component (Spray), facilities for handling {\sc
  rest} requests. We chose the {\sc amqp} protocol and its Rabbit{\sc
  mq} implementation for implementing the request queue because recent
versions include support for active/active cluster configurations. The
persistence layer is currently implemented by Mongo{\sc db}, for its
replication and sharding support. However, this is not a hard
requirement, as Aquarium features an abstraction layer for all
database queries (currently 10 methods), which can then be implemented
by any persistence system, relational or not.


\section{Performance}

To evaluate the performance of Aquarium, we formulated an experiment
that evaluated two important properties: the time required to perform
the charging operation for a resource event and the overall time
required to process a resource event, end to end. To conduct the
experiment, Aquarium was configured using the policy {\sc dsl} to
handle billing events for 5 types of resources, using 3 overloaded
price lists, 2 overloaded algorithms, all of which were combined to 5
different agreements. Aquarium's data store was pre-filled in with
1,000,000 resource events, evenly distributed among 1,000 users. To
drive the benchmark we used a synthetic load generator that produced
random billing events, at a configurable rate per minute.

To run the benchmark we deployed Aquarium on a virtualized 4 core
2{\sc gh}z class {\sc cpu} and 4{\sc gb} {\sc ram} Debian Linux server.
The virtual machine running Aquarium was configured with a 4{\sc gb}
maximum heap size. Rabbit{\sc mq} and Mongo{\sc db} where run in
another 4-core, 4{\sc gb ram} virtual machine. The two virtual
machines did not share a physical host and communicated over the
infrastructure's switched network fabric at an effective rate of 800
Mbits/sec, as reported by the \texttt{iperf} utility. No further
optimization was performed on either back-end system.

\begin{figure}[t]
    \begin{center}
        \includegraphics[scale=0.59]{perf.pdf}
    \end{center}

    \caption{Average time for performing a billing operation and end
      for end to end message processing for 100 active users and a
      varying number of messages per minute.}
    
    \label{fig:perf}
\end{figure}

All measurements were done using the first working version of the
Aquarium deployment, so no real optimization effort has taken place.
This shows in the current performance measurements, as Aquarium was
not able to handle more than about 500 billing operations per second
(see Figure~\ref{fig:perf}). One factor that contributed to this result
was the way resource state recalculations was done; in the current
version, the system needs to re-read parts of the event and billing
state from the datastore every time a new resource event appears. This
contributes to more than 50\% of the time required to produce a
charging event, and can be completely eliminated when proper billing
snapshots are implemented. In other measurements, we also observed
that the rate of garbage creation was extremely high, more that 250
{\sc mb}/sec. Upon further investigation, we attributed it to the way
policy timeslot applicability is calculated. Despite the high
allocation rate, the {\sc jvm}'s garbage collector never went through
a full collection cycle; when we forced one after the benchmark run
was over, we observed that the actual heap memory usage was only
80{\sc mb}, which amounts to less than 1 {\sc mb} per user.

% Even so, by extrapolating on the results and hardware configuration, an average
% 12-core box could handle more 1.500 messages per minute from about 300 active
% users, at 5 events per minute. Given that activity from users is expected to
% arrive in bursts, a more realistic expectation might be to receive 1 message
% per user per 5 minutes on average; in that case, and provided that the resource
% query cost is negligible as it is being served from an in memory cache, the
% system could handle around 4.500 concurrent users. While such back of the
% envelop calculations do not account for traffic spikes, they do provide an
% rough estimation of the optimisation effort that must be put in place. 

\section{Related Work}

Yousef et al.~\cite{Youse08} described the three pricing models that
are used by cloud service providers for billing used resources, namely
tiered pricing, per-unit pricing and subscription-based pricing.
Aquarium's cost policies that are assigned to resources map exactly to
Yousef's pricing models. In fact, most offerings by public IaaS
providers, including Amazon and Azure, offer services charged
according to Yousef models.

Work on resource accounting and billing has been carried out in the
context of cloud federation~\cite{Rochw09, Elmro09} and (earlier) grid
federation projects. The Distributed Grid Accounting System ({\sc
  dgas})~\cite{Piro06} was among the first to enable resource
accounting at the computing and storage layers and then aggregation of
the resources in a centralized location. The Reservoir project
investigated the use of service level agreements~\cite{Elmro09} for
resource provisioning in federated cloud scenarios.

On the cloud computing front, vendors such as VMWare, Microsoft and
{\sc ibm} provide full stack solutions, which also include resource
accounting. Usually, such systems are connected with existing
enterprise resource planning systems. \"Ubersmith has developed an
engine dedicated to resource accounting; much like Aquarium, it tracks
resource usage and applies accounting policies to it. On the open
source front, neither the Cloustack nor the Openstack projects, have
yet incorporated billing into their services. To the best of our
knowledge, Aquarium is the first open source system to offer
configurable accounting services for IaaS deployments.

%Check also~\cite{Ruiz-Agundez11}.

\section{Lessons Learned and Future Work}

A typesafe language was a hard requirement. Of the platforms examined,
the {\sc jvm} had the richest collection of ready made components; the
Akka library was particularly enticing for the scalability and
distribution possibilities it offered.

Scala as a language was an enabling factor; case classes permitted the
expression of data models, including the configuration {\sc dsl}, that
could be easily be serialized or read back from wire formats while
also promoting immutability through the use of the \texttt{copy()}
constructor. The pervasive use of immutability allowed us to write
strict, yet simple and concise unit tests, as the number of cases to
be examined was generally low. 

Akka's custom supervision hierarchies allowed us to partition the
system in self-healing sub-components, each of which can fail
independently of the other. For example, if the queue reader component
fails due to a queue failure, Aquarium will still be accessible and
responsive for the {\sc rest} interface. Also, Akka allowed us to
easily saturate the processing components of any system we tested
Aquarium on, simply by tuning the number of threads (in {\sc i/o}
bound parts) and actors (in {\sc cpu} bound parts) per dispatcher.

From a software engineering point of view, the current state of the
project was reached using about 8 person months of effort, 2 of which
were devoted to requirements elicitation, prototype building and
familiarizing with the language. The source code currently consists of
7,000 lines of executable statements (including about 1,200 lines of
tests), divided in about 10 packages. The system is built using both
{\sc sbt} and Maven. In the future we will add a comprehensive {\sc
  rest api} for accessing the user actor state and we will distribute
the message processing across multiple nodes in an active-active mode.

\section{Acknowledgments and Availability}

Aquarium is developed under a contract from the Greek Ministry of
Education for the ``Advanced Computing Services for the Research and
Academic Community'' project (code 296114) and is available under a {\sc bsd}
2-clause license from: \url{https://code.grnet.gr/projects/aquarium}.

{\footnotesize \bibliographystyle{acm}
\bibliography{aquarium}}


\end{document}
